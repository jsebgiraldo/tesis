\chapter{Introducción}

\textit{Este capítulo establece el contexto y la motivación de la investigación, presentando los desafíos actuales de las redes eléctricas inteligentes (Smart Energy) en la era de la transición energética. Se analizan las limitaciones de las arquitecturas tradicionales basadas en la nube, se comparan las principales tecnologías de comunicación IoT disponibles (Thread, Zigbee, Bluetooth Mesh, LoRaWAN, Wi-Fi HaLow), y se justifica la elección de la arquitectura propuesta. El capítulo plantea el problema de investigación, delimita el alcance del trabajo, formula las hipótesis a validar y establece los objetivos generales y específicos. Finalmente, se describe la estructura del documento y la metodología empleada para el desarrollo de la tesis.}

\section{Contexto y Motivación}

\subsection{El Desafío de las Redes Smart Energy}

La transición energética global hacia sistemas descentralizados, con alta penetración de energías renovables distribuidas (DER, por sus siglas en inglés \textit{Distributed Energy Resources}) y gestión activa de la demanda (DSM, \textit{Demand Side Management}), exige infraestructuras de medición inteligente robustas y escalables~\cite{velasquezSmartGridsEmpowered2024,SmartHomeEnergy2024}. Estas infraestructuras, conocidas como AMI (\textit{Advanced Metering Infrastructure}), deben ser capaces de recolectar, transmitir y procesar datos de millones de puntos de consumo en tiempo cuasi-real, proporcionando la información necesaria para optimizar la operación de la red eléctrica~\cite{alsafranChallengesImplementingIoT2025}.

Según proyecciones de la Agencia Internacional de Energía (IEA, \textit{International Energy Agency}), se anticipa la instalación de más de 1.300 millones de medidores inteligentes a nivel global para el año 2030. Este despliegue masivo generará aproximadamente 15 petabytes (PB) de datos de telemetría diarios, planteando desafíos significativos en términos de comunicación, almacenamiento y procesamiento de información~\cite{dianeSystematicComprehensiveReview2025}.

Sin embargo, las arquitecturas tradicionales basadas en comunicación directa dispositivo-nube enfrentan limitaciones críticas que comprometen su viabilidad técnica y económica. En primer lugar, estas soluciones presentan latencias elevadas (superiores a 200 milisegundos), lo que dificulta aplicaciones de tiempo real como la respuesta a la demanda. Además, exhiben una dependencia estricta de conectividad WAN (\textit{Wide Area Network}) continua, generando vulnerabilidad ante interrupciones del servicio de internet. Por otra parte, los costos operacionales se vuelven prohibitivos en escenarios de alta densidad de dispositivos, debido al alto consumo de ancho de banda y los cargos por transferencia de datos a la nube. Finalmente, estas arquitecturas presentan dificultades para garantizar los requisitos de tiempo real exigidos por aplicaciones críticas como la gestión de microrredes y la respuesta automatizada a la demanda (DR, \textit{Demand Response}).

\subsection{Estado Actual de las Tecnologías de Comunicación IoT}

Para abordar los desafíos planteados en la sección anterior, es fundamental comprender el panorama actual de las tecnologías de comunicación disponibles para aplicaciones IoT (\textit{Internet of Things}) en el sector energético~\cite{abdulsalamOverviewRecentWireless2024,choudharyInternetThingsComprehensive2024}. El ecosistema IoT para aplicaciones industriales y de infraestructura crítica se caracteriza por una heterogeneidad de tecnologías de comunicación, cada una optimizada para rangos específicos de alcance, throughput (capacidad de transmisión), latencia y consumo energético~\cite{ashfaqIoTSensorNetworks2024}. Esta diversidad tecnológica permite seleccionar la combinación más adecuada según los requisitos específicos de cada aplicación y escenario de despliegue.

A continuación, se presenta una comparativa técnica de las principales tecnologías de comunicación relevantes para redes de medición inteligente, agrupadas en tres categorías: protocolos mesh de corto alcance (2.4 GHz), plataformas de procesamiento en el borde (edge computing), y tecnologías de última milla para conectividad de área amplia.

\subsubsection{Comparativa Técnica de Protocolos Mesh 2.4 GHz}

\begin{table}[h]
\centering
\small
\caption{Comparación de protocolos mesh 2.4 GHz para IoT (Thread, Zigbee, Bluetooth Mesh)}
\label{tab:mesh-comparison}
\begin{tabular}{|p{3.2cm}|p{3.5cm}|p{3.5cm}|p{3.5cm}|}
\hline
\rowcolor{gray!20}
\textbf{Característica} & \textbf{Thread 1.3.1} & \textbf{Zigbee 3.0} & \textbf{Bluetooth Mesh} \\
\hline
\textbf{Capa física} & IEEE 802.15.4 & IEEE 802.15.4 & Bluetooth 5.3 LE \\
\hline
\textbf{Frecuencia} & 2.4 GHz & 2.4/Sub-GHz & 2.4 GHz \\
\hline
\textbf{Topología} & Mesh (MLE routing) & Mesh (AODV) & Managed Flooding \\
\hline
\textbf{IPv6 nativo} & \textcolor{blue}{Sí} (6LoWPAN) & \textcolor{red}{No} (propietario) & \textcolor{red}{No} (GATT proxy) \\
\hline
\textbf{Nodos máx.} & $>$250 & 65,535 (teórico) & 32,767 \\
\hline
\textbf{Latencia (3 hops)} & \textcolor{blue}{40-60 ms} & 80-120 ms & 100-200 ms \\
\hline
\textbf{Consumo RX/TX} & 19/22 mA & 24/31 mA & \textcolor{blue}{9.2/10.5 mA} \\
\hline
\textbf{Sleep current} & 5 µA (ESP32-C6) & 10 µA típico & \textcolor{blue}{2 µA} (nRF52840) \\
\hline
\textbf{Interoperabilidad} & \textcolor{blue}{OTBR estándar} & Req. coordinador & Req. provisioner \\
\hline
\textbf{Seguridad} & TLS/DTLS 1.2 & AES-128 CCM & AES-CCM \\
\hline
\end{tabular}
\end{table}

Como se observa en la Tabla \ref{tab:mesh-comparison}, Thread emerge como el protocolo preferencial para redes de campo en aplicaciones de Smart Energy debido a tres ventajas fundamentales. En primer lugar, su routing IPv6 nativo facilita la integración con infraestructuras IP existentes, eliminando la necesidad de gateways de traducción de protocolo propietarios. En segundo lugar, cuenta con una estandarización completa bajo Thread Group (miembro de la Connectivity Standards Alliance), lo que garantiza interoperabilidad entre fabricantes. Finalmente, ofrece soporte multi-vendor certificado mediante el programa de certificación Thread 1.3.1, reduciendo el riesgo de vendor lock-in en proyectos de largo plazo.

Además, Thread presenta latencias significativamente menores (40-60 ms en tres saltos) comparado con Zigbee (80-120 ms) y Bluetooth Mesh (100-200 ms), lo cual resulta crítico para aplicaciones que requieren respuesta en tiempo real, como la detección de anomalías en el consumo eléctrico o la coordinación de microrredes.

\subsubsection{Plataformas de Edge Computing - Análisis Comparativo}

\begin{table}[h]
\centering
\small
\caption{Comparación de plataformas edge IoT para procesamiento distribuido}
\label{tab:edge-platforms}
\begin{tabular}{|p{2.8cm}|p{2.8cm}|p{2.8cm}|p{2.8cm}|p{2.8cm}|}
\hline
\rowcolor{gray!20}
\textbf{Plataforma} & \textbf{ThingsBoard Edge} & \textbf{AWS Greengrass} & \textbf{Azure IoT Edge} & \textbf{Node-RED} \\
\hline
\textbf{Arquitectura} & Monolítica Java & Microservices Python & Containerizada .NET & Flow-based JS \\
\hline
\textbf{Sincronización} & \textcolor{blue}{Bidireccional} & \textcolor{red}{Unidireccional} & \textcolor{blue}{Bidireccional} & Manual \\
\hline
\textbf{Rule Engine local} & \textcolor{blue}{Sí} (full chain) & Lambda local & Módulos custom & Function nodes \\
\hline
\textbf{Almacenamiento} & PostgreSQL/Cassandra & DynamoDB local & SQLite/Custom & Context store \\
\hline
\textbf{Dashboard local} & \textcolor{blue}{Sí} (full featured) & \textcolor{red}{No} (CloudWatch) & \textcolor{red}{No} (portal cloud) & \textcolor{blue}{UI integrado} \\
\hline
\textbf{Autonomía offline} & \textcolor{blue}{Ilimitada} & Limitada & Limitada & \textcolor{blue}{Ilimitada} \\
\hline
\textbf{Footprint RAM} & 1-4 GB & 512 MB-2 GB & 256 MB-1 GB & \textcolor{blue}{128-512 MB} \\
\hline
\textbf{Licenciamiento} & \textcolor{blue}{Apache 2.0} & \textcolor{red}{Propietario} & \textcolor{red}{Propietario} & \textcolor{blue}{Apache 2.0} \\
\hline
\textbf{Curva aprendizaje} & Media & Alta & Alta & \textcolor{blue}{Baja} \\
\hline
\end{tabular}
\end{table}

Del análisis comparativo presentado en la Tabla \ref{tab:edge-platforms}, ThingsBoard Edge se posiciona como la solución más robusta para aplicaciones industriales que requieren continuidad operacional durante particiones WAN prolongadas. A diferencia de las alternativas comerciales propietarias (AWS IoT Greengrass, Azure IoT Edge), ThingsBoard Edge proporciona capacidades completas de procesamiento de reglas (rule engine), dashboards interactivos accesibles localmente y sincronización bidireccional de configuraciones y datos históricos. 

Esta autonomía offline ilimitada resulta especialmente relevante en el contexto latinoamericano, donde las infraestructuras de telecomunicaciones pueden presentar interrupciones frecuentes, particularmente en zonas rurales y semi-urbanas. Adicionalmente, su licenciamiento Apache 2.0 elimina costos recurrentes de suscripción y permite personalización del código fuente según requisitos específicos del proyecto.

\subsubsection{HaLow - Posicionamiento frente a Alternativas de Última Milla}

\begin{table}[h]
\centering
\small
\caption{Comparación de tecnologías última milla para Smart Energy}
\label{tab:lastmile-comparison}
\begin{tabular}{|p{3.2cm}|p{2.8cm}|p{2.5cm}|p{2.5cm}|p{2.5cm}|}
\hline
\rowcolor{gray!20}
\textbf{Característica} & \textbf{HaLow 802.11ah} & \textbf{LoRaWAN} & \textbf{LTE Cat-M1} & \textbf{Wi-Fi 6} \\
\hline
\textbf{Frecuencia} & Sub-GHz (900 MHz) & Sub-GHz (868/915) & LTE Bands & 2.4/5 GHz \\
\hline
\textbf{Alcance típico} & \textcolor{blue}{1-2 km} & 5-15 km & 10-35 km & \textcolor{red}{50-100 m} \\
\hline
\textbf{Throughput máx.} & \textcolor{blue}{40 Mbps} (4 MHz) & \textcolor{red}{50 kbps} & 1 Mbps & 9.6 Gbps \\
\hline
\textbf{Latencia típica} & \textcolor{blue}{10-30 ms} & \textcolor{red}{1-5 seg} & 50-100 ms & <10 ms \\
\hline
\textbf{Topología} & \textcolor{blue}{Star/Mesh} & Star (sin mesh) & Star (celular) & Star \\
\hline
\textbf{Consumo TX (avg)} & 180 mA @ 1 MHz & \textcolor{blue}{120 mA} & 220 mA & 350 mA \\
\hline
\textbf{Cobertura indoor} & \textcolor{blue}{Excelente} (penetración) & Media & Buena & Limitada \\
\hline
\textbf{Espectro} & \textcolor{blue}{No licenciado} ISM & \textcolor{blue}{No licenciado} ISM & \textcolor{red}{Licenciado} (operador) & \textcolor{blue}{No licenciado} ISM \\
\hline
\textbf{Despliegue} & Privado (CAPEX) & Gateway privado & \textcolor{red}{Suscripción} MVNO & Privado (CAPEX) \\
\hline
\textbf{Costo por nodo} & \$25-40 módulo & \textcolor{blue}{\$8-15} módulo & \$12-25 módulo & \textcolor{blue}{\$5-10} módulo \\
\hline
\end{tabular}
\end{table}

Como se evidencia en la Tabla \ref{tab:lastmile-comparison}, Wi-Fi HaLow (IEEE 802.11ah) combina las ventajas de diferentes tecnologías de última milla en un único estándar. Frente a LoRaWAN, ofrece un throughput superior (40 Mbps vs 50 kbps), lo que permite la transmisión de datos agregados de múltiples medidores sin congestión. Comparado con LTE Cat-M1, proporciona latencia determinística menor (10-30 ms vs 50-100 ms) y elimina los costos recurrentes de suscripción a operadores móviles (MVNO, \textit{Mobile Virtual Network Operator}). Por otra parte, supera significativamente al Wi-Fi 6 convencional en alcance (1-2 km vs 50-100 m) gracias a su operación en bandas sub-GHz con mayor capacidad de penetración en edificaciones.

Adicionalmente, HaLow opera en espectro no licenciado ISM (\textit{Industrial, Scientific and Medical}), permitiendo despliegues privados controlados por el operador de la red eléctrica sin dependencia de infraestructura de terceros. Esta característica posiciona a Wi-Fi HaLow como la tecnología óptima para el backhaul de gateways Smart Energy en zonas urbanas y suburbanas de densidad media-alta, donde se requiere un balance entre alcance, capacidad y autonomía operativa.

\subsection{Brechas en Arquitecturas IoT Existentes}

A pesar de los avances tecnológicos descritos en las secciones anteriores, el análisis crítico del estado del arte revela limitaciones estructurales en las arquitecturas IoT contemporáneas que impiden su adopción masiva en aplicaciones de infraestructura crítica como las redes eléctricas inteligentes. Estas brechas se manifiestan en tres dimensiones principales: dependencia excesiva de conectividad cloud, ineficiencias en la utilización del ancho de banda y ausencia de capacidades de procesamiento inteligente distribuido.

\begin{itemize}
\item \textbf{Dependencia cloud-centric}: Las arquitecturas tradicionales dispositivo → cloud presentan Single Points of Failure (SPOF) en enlaces WAN. Estudios empíricos en despliegues urbanos reportan disponibilidades de 94-96\% en conectividad celular LTE (downtimes acumulados 18-25 días/año), insuficientes para aplicaciones críticas.

\item \textbf{Overhead de traducción multi-protocolo}: Los gateways convencionales implementan traductores application-layer (ej. Thread → MQTT → HTTP → Cloud), introduciendo latencias acumuladas de 150-300 ms y complejidad en mantenimiento de mapeos de datos.

\item \textbf{Escalabilidad limitada del cloud ingestion}: Plataformas cloud IoT típicamente cobran por mensaje ingestado (\$5-10 por millón de mensajes), resultando en costos prohibitivos para aplicaciones de telemetría de alta frecuencia (ej. 10,000 medidores reportando cada 5 minutos generan \$2,880/mes solo en ingesta).

\item \textbf{Ausencia de estándares de interoperabilidad}: La mayoría de soluciones comerciales implementan APIs propietarias, dificultando la migración entre vendors y bloqueando clientes en ecosistemas cerrados.
\end{itemize}

\subsubsection{Análisis Cuantitativo de Overhead en Arquitecturas Tradicionales}

\begin{table}[h]
\centering
\small
\caption{Latencia end-to-end por arquitectura (device → cloud storage)}
\label{tab:latency-overhead}
\begin{tabular}{|p{3.8cm}|p{3.2cm}|p{3.2cm}|p{3.2cm}|}
\hline
\rowcolor{gray!20}
\textbf{Componente} & \textbf{Cloud-Centric} & \textbf{Edge-Lite (Node-RED)} & \textbf{Propuesta (Edge Full)} \\
\hline
\textbf{Device → Gateway} & 40 ms (Thread) & 40 ms (Thread) & 40 ms (Thread) \\
\hline
\textbf{Gateway → WAN} & 80 ms (LTE) & 15 ms (Ethernet) & 15 ms (HaLow/Eth) \\
\hline
\textbf{WAN → Cloud} & 50 ms (RTT) & 50 ms (RTT) & \textcolor{blue}{N/A} (local) \\
\hline
\textbf{Cloud processing} & 30 ms (ingestion) & 30 ms (ingestion) & \textcolor{blue}{N/A} \\
\hline
\textbf{Cloud → DB write} & 10 ms (RDS write) & 10 ms (RDS write) & \textcolor{blue}{8 ms} (TimescaleDB) \\
\hline
\rowcolor{yellow!20}
\textbf{\textbf{TOTAL P50}} & \textbf{210 ms} & \textbf{145 ms} & \textbf{\textcolor{blue}{63 ms}} \\
\hline
\rowcolor{yellow!20}
\textbf{\textbf{TOTAL P99}} & \textbf{450 ms} & \textbf{310 ms} & \textbf{\textcolor{blue}{95 ms}} \\
\hline
\end{tabular}
\end{table}

La arquitectura propuesta reduce latencia end-to-end en 70\% (P50) y 79\% (P99) respecto a arquitecturas cloud-centric, eliminando el round-trip WAN mediante procesamiento local completo.

\section{Planteamiento del Problema}

\subsection{Definición del Problema de Investigación}

Las redes de telemetría para Smart Energy enfrentan limitaciones críticas en sus arquitecturas de comunicación que comprometen la eficiencia operacional y escalabilidad de los sistemas de gestión energética inteligente. Estas limitaciones se manifiestan en tres dimensiones interrelacionadas:

\textbf{Problema 1 - Overhead excesivo en protocolos de comunicación}: Las arquitecturas tradicionales de telemetría energética utilizan protocolos no optimizados para dispositivos con restricciones de recursos (MQTT/JSON sobre TCP/IP), generando overhead de paquetes que alcanza 60-80\% del frame total en redes de sensores IEEE 802.15.4 con MTU de 127 bytes. Un paquete típico MQTT/JSON con lectura de consumo energético (payload útil 15-20 bytes) transporta 48 bytes de headers IPv6+UDP+TCP+MQTT, resultando en eficiencia de transmisión <30\%. Este overhead se amplifica en topologías mesh multi-salto, donde cada retransmisión replica headers completos, generando latencias acumuladas de 150-300 ms en rutas de 3-5 saltos y consumo energético excesivo que reduce vida útil de baterías de 5 años proyectados a 18-24 meses reales en nodos alimentados por batería.

La ausencia de mecanismos estandarizados de compresión de headers IPv6 y optimización de protocolos de aplicación para redes constrained impide alcanzar los requisitos de eficiencia espectral y latencia determinística exigidos por aplicaciones críticas de gestión de demanda (demand response) y coordinación de recursos energéticos distribuidos (DER), donde ventanas de respuesta de 50-100 ms son mandatorias según estándares IEEE 2030.5 y IEC 61850-90-5.

\textbf{Problema 2 - Dependencia crítica de conectividad WAN continua}: Las arquitecturas cloud-centric tradicionales (dispositivo → gateway → WAN → cloud) presentan Single Points of Failure en enlaces de área amplia, con disponibilidades reportadas de 94-96\% en conectividad celular LTE en despliegues urbanos (equivalente a 15-22 días de downtime anual). Durante particiones WAN, los sistemas pierden capacidades críticas: visualización de telemetría en tiempo real para operadores, ejecución de reglas de negocio (alarmas, eventos), persistencia de datos históricos, y gestión remota de dispositivos. Esta dependencia genera riesgos operacionales en infraestructuras críticas donde continuidad de servicio es mandatoria.

La arquitectura centralizada introduce además latencias estructurales inherentes (device → gateway: 40 ms Thread, gateway → WAN: 80 ms LTE, WAN → cloud: 50 ms RTT, cloud processing: 30 ms, cloud → DB: 10 ms) que acumulan 210 ms en percentil P50 y >450 ms en P99, excediendo requisitos de aplicaciones de respuesta rápida a la demanda (<100 ms) y coordinación de microrredes (<50 ms). La imposibilidad de procesamiento local durante desconexiones WAN impide implementar estrategias de gestión autónoma de energía en escenarios de islanding de microrredes.

\textbf{Problema 3 - Limitaciones de alcance y throughput en tecnologías de última milla}: Las tecnologías de comunicación predominantes para backhaul de gateways Smart Energy presentan trade-offs desfavorables. LoRaWAN ofrece alcance extendido (5-15 km) pero throughput extremadamente limitado (50 kbps máximo, 0.3-50 kbps típico) y latencias impredecibles (1-5 segundos), inadecuadas para aplicaciones de telemetría de alta frecuencia (lecturas cada 5-15 minutos) y comandos de control en tiempo real. LTE Cat-M1 proporciona throughput superior (1 Mbps) y latencia aceptable (50-100 ms) pero genera costos operacionales recurrentes significativos (\$10-15 USD por nodo por año) que en despliegues de 1,000+ medidores resultan en OPEX prohibitivos (\$150,000 en 5 años solo en conectividad), además de requerir cobertura celular que puede ser intermitente en zonas suburbanas y rurales.

Wi-Fi tradicional 2.4/5 GHz ofrece alto throughput pero alcance limitado (50-100 m) y pobre penetración en entornos NLOS (Non-Line-of-Sight), requiriendo despliegue denso de puntos de acceso con CAPEX elevado. La ausencia de tecnologías que combinen alcance extendido (>1 km), throughput suficiente para agregación de datos (>40 Mbps), latencia determinística (<50 ms), y operación en espectro no licenciado sin costos recurrentes, limita la viabilidad económica de redes de telemetría de gran escala.

\textbf{Impacto del problema}: Estas limitaciones resultan en sistemas de telemetría Smart Energy con eficiencia operacional subóptima, costos de propiedad (TCO) elevados, escalabilidad restringida, y dependencia de conectividad externa que compromete resiliencia ante fallos. La ausencia de estándares abiertos de interoperabilidad agrava el problema, generando lock-in tecnológico y dificultando integración multi-vendor.

\subsection{Delimitación del Problema}

El problema de investigación se delimita específicamente al contexto de **redes de telemetría Smart Energy basadas en 6LoWPAN** para monitoreo y gestión de consumo energético en infraestructuras de distribución eléctrica residencial y comercial. La delimitación se estructura en tres dimensiones:

\textbf{Dimensión 1 - Dominio de Aplicación: Smart Energy}

El problema se circunscribe exclusivamente a aplicaciones de **gestión inteligente de energía eléctrica** según estándares IEEE 2030.5 (Smart Energy Profile 2.0) e IEC 61850, enfocándose en:

\begin{itemize}
\item \textbf{Telemetría de consumo}: Recolección de datos de medidores inteligentes (smart meters) con frecuencias de muestreo de 5-60 minutos, incluyendo mediciones de potencia activa/reactiva (kW/kVAr), voltaje (V), corriente (A), factor de potencia, y energía acumulada (kWh).
\item \textbf{Gestión de demanda (Demand Response)}: Comunicación bidireccional para implementación de eventos de respuesta a la demanda (DR) con ventanas de respuesta de 50-100 ms, incluyendo señalización de precios dinámicos, control de cargas, y participación en mercados de flexibilidad.
\item \textbf{Monitoreo de calidad de energía}: Detección de sags/swells de voltaje, interrupciones, armónicos, y eventos de calidad de potencia según IEC 61000-4-30.
\item \textbf{Integración de recursos energéticos distribuidos (DER)}: Coordinación de generación solar fotovoltaica, almacenamiento en baterías, vehículos eléctricos, y gestión de microrredes con requisitos de latencia <50 ms para sincronización de fasores.
\end{itemize}

Se excluyen del alcance: telemetría de agua/gas, monitoreo industrial (no energético), automatización de edificios (HVAC, iluminación no vinculada a gestión energética), y sistemas SCADA de alta tensión en subestaciones (dominio de IEC 61850-3).

\textbf{Dimensión 2 - Stack de Protocolos: 6LoWPAN como Capa de Adaptación}

El problema se enfoca en la **optimización de comunicaciones mediante 6LoWPAN** (RFC 6282, RFC 4944) como capa de adaptación IPv6 para redes de sensores con restricciones de recursos, delimitando:

\begin{itemize}
\item \textbf{Capa física/MAC}: IEEE 802.15.4-2020 banda 2.4 GHz, OQPSK modulation, 250 kbps, MTU 127 bytes, CSMA/CA con backoff exponencial.
\item \textbf{Capa de adaptación (6LoWPAN)}: Compresión IPHC (IPv6 Header Compression) reduciendo headers de 40 bytes a 2-7 bytes, compresión NHC (Next Header Compression) para UDP/TCP, fragmentación y reensamblado para paquetes >127 bytes, mesh-under routing con headers de encapsulación.
\item \textbf{Capa de transporte}: UDP predominante (overhead 8 bytes comprimible a 4 bytes con NHC), TCP limitado para aplicaciones que requieren confiabilidad garantizada (ej. firmware updates).
\item \textbf{Capa de aplicación}: CoAP (Constrained Application Protocol, RFC 7252) como protocolo RESTful ligero con overhead 4-10 bytes, modos CON/NON, Observe (RFC 7641) para subscripciones, block-wise transfer (RFC 7959) para transferencias grandes, y DTLS 1.2 para seguridad.
\item \textbf{Gestión de dispositivos}: LwM2M 1.2 (Lightweight M2M, OMA SpecWorks) sobre CoAP, con objetos estándar para telemetría energética, firmware OTA, y monitoreo de conectividad.
\end{itemize}

El problema se delimita a la evaluación cuantitativa de: (a) reducción de overhead de paquetes mediante compresión 6LoWPAN vs stacks tradicionales MQTT/TCP, (b) latencia por salto en topologías mesh Thread de 3-5 hops, (c) eficiencia energética (mJ/bit) en nodos alimentados por batería, y (d) packet delivery ratio (PDR) en condiciones de interferencia 2.4 GHz.

\textbf{Dimensión 3 - Alcance Geográfico y Escala}

\begin{itemize}
\item \textbf{Entorno de despliegue}: Zonas urbanas y suburbanas residenciales/comerciales con densidades de 100-500 medidores por km², excluyendo zonas rurales remotas (baja densidad <20 medidores/km²) y zonas industriales de alta potencia (>1 MW por punto de medición).
\item \textbf{Escala de red}: Topologías de 10-100 nodos IoT por gateway edge, con validación experimental en prototipo de 10 nodos y extrapolación analítica a 100 nodos. Se excluye la validación empírica de redes >1,000 nodos.
\item \textbf{Alcance de comunicación}: Redes Thread mesh con alcance efectivo 200-500 m (3-5 hops @ 80 m por hop en entorno urbano con obstrucciones), y backhaul HaLow con alcance 1-2 km en configuración 2 MHz bandwidth.
\item \textbf{Requisitos temporales}: Latencia end-to-end objetivo <100 ms P95 para telemetría, <50 ms para comandos de control demand response, y disponibilidad >99\% anual (downtime <87 horas/año).
\end{itemize}

\textbf{Estándares implementados}:
\begin{itemize}
\item \textbf{Smart Energy}: IEEE 2030.5-2023 (Function Sets: DCAP, Time, EndDevice, MirrorUsagePoint, MirrorMeterReading), ISO/IEC 30141:2024 (IoT Reference Architecture).
\item \textbf{Comunicación 6LoWPAN}: RFC 6282 (IPHC), RFC 4944 (6LoWPAN), RFC 7252 (CoAP), RFC 7641 (Observe), RFC 7959 (Block-wise), OMA LwM2M 1.2.
\item \textbf{Conectividad}: IEEE 802.15.4-2020 (Thread 1.3.1), IEEE 802.11ah-2016 (HaLow).
\end{itemize}

\textbf{Exclusiones explícitas}: PLC (Power Line Communication G3-PLC/PRIME), protocolos propietarios (Zigbee Smart Energy 1.x), redes celulares 5G/NR-Light, redes de alta tensión con IEC 61850-3 (fuera del dominio Smart Energy residencial/comercial), y blockchain para auditoría de transacciones energéticas (trabajo futuro).

Esta delimitación asegura que el problema de investigación se mantenga enfocado en la intersección específica de **6LoWPAN como solución de comunicación eficiente** y **Smart Energy como dominio de aplicación crítico**, evitando dispersión en dominios adyacentes que diluirían la contribución técnica.

\subsection{Justificación}

\subsubsection{Justificación Técnica}

Las arquitecturas edge-computing para IoT industrial requieren capacidades de procesamiento local, almacenamiento persistente y autonomía operacional que las soluciones cloud-centric tradicionales no pueden garantizar. La integración de Wi-Fi HaLow como tecnología de backhaul representa una innovación técnica respecto al estado del arte (dominado por LTE/LoRaWAN), aprovechando sus ventajas de throughput (40 Mbps vs 1 Mbps LTE Cat-M1), latencia (<30 ms vs >50 ms), y ausencia de costos recurrentes de conectividad.

\subsubsection{Justificación Económica}

Análisis de TCO (Total Cost of Ownership) para despliegue de 1,000 puntos de medición durante 5 años:

\begin{itemize}
\item \textbf{Cloud-centric + LTE:} CAPEX \$150k (hardware) + OPEX \$180k (conectividad \$15/nodo/año) = \$330k
\item \textbf{Propuesta HaLow:} CAPEX \$200k (hardware + APs HaLow) + OPEX \$25k (mantenimiento) = \$225k
\item \textbf{Ahorro proyectado:} 32\% (\$105k en 5 años)
\end{itemize}

\subsubsection{Justificación Académica}

La investigación contribuye al cuerpo de conocimiento en arquitecturas IoT heterogéneas mediante:
\begin{itemize}
\item Diseño de arquitectura de referencia para gateways multi-PHY conformes con ISO/IEC 30141.
\item Caracterización empírica de latencias en integración Thread ↔ HaLow.
\item Metodología de implementación de IEEE 2030.5 Function Sets sobre plataformas embebidas Linux.
\item Evaluación comparativa de estrategias de failover multi-WAN en gateways IoT.
\end{itemize}

\subsection{Metodología de Investigación}

La investigación sigue un enfoque mixto que combina Design Science Research (DSR) para el diseño de artefactos tecnológicos, Investigación Experimental para la validación de hipótesis cuantitativas, y Estudio de Caso para la evaluación en contexto real.

\subsubsection{Fase 1 - Análisis y Diseño (Design Science)}

\textbf{Objetivos:} Especificar requisitos funcionales/no funcionales, diseñar arquitectura de referencia multi-capa, definir interfaces entre componentes.

\textbf{Actividades:}
\begin{enumerate}
\item Revisión sistemática de literatura sobre arquitecturas IoT edge y estándares Smart Energy (IEEE 2030.5, ISO/IEC 30141, IEC 61850).
\item Análisis comparativo de tecnologías de comunicación (Thread, Zigbee, BLE Mesh, HaLow, LoRaWAN, LTE Cat-M1).
\item Diseño de arquitectura de 4 capas: Conectividad, Orquestación, Procesamiento, Aplicación.
\item Especificación de interfaces: OTBR APIs, MQTT topics, IEEE 2030.5 REST endpoints.
\item Modelado de latencias mediante teoría de colas (M/M/1 para gateway, M/G/∞ para cloud).
\end{enumerate}

\textbf{Entregables:} Diagrama de arquitectura (Capítulo 3), especificación de requisitos (Capítulo 3.3), diseño de base de datos TimescaleDB (Anexo B).

\subsubsection{Fase 2 - Implementación (Engineering)}

\textbf{Objetivos:} Implementar gateway prototipo funcional, integrar componentes hardware/software, desarrollar servicios containerizados.

\textbf{Actividades:}
\begin{enumerate}
\item Configuración plataforma hardware: Banana Pi BPI-R4 (4x Cortex-A53 @ 1.8 GHz, 4 GB RAM) + nRF52840 RCP (Thread) + Morse Micro MM6108 (HaLow) + Quectel EG25-G (LTE).
\item Instalación y configuración OpenWRT 23.05.x con kernel real-time patches (PREEMPT\_RT).
\item Despliegue stack Docker Compose: ThingsBoard Edge 3.6.0, PostgreSQL 15 + TimescaleDB 2.13, Apache Kafka 7.5.0, IEEE 2030.5 Server (Python/Flask), Ollama LLM (Llama 3.2 3B).
\item Implementación IEEE 2030.5 Function Sets: DCAP, Time, EndDevice, MirrorUsagePoint, MirrorMeterReading, Messaging (XML schemas según estándar).
\item Configuración mwan3 para failover multi-WAN (Ethernet métrica 10, HaLow STA métrica 15, LTE métrica 20).
\item Desarrollo nodos IoT: ESP32-C6 Thread LwM2M + sensor BME280 (temperatura/humedad/presión).
\end{enumerate}

\textbf{Entregables:} Documentación de instalación (Anexo A), archivos docker-compose.yml (Anexo B), scripts de integración (Anexo C), código fuente nodos IoT (Anexo E).

\subsubsection{Fase 3 - Validación Experimental}

\textbf{Objetivos:} Validar hipótesis mediante mediciones empíricas, caracterizar rendimiento del sistema, evaluar resiliencia ante fallos.

\textbf{Experimentos:}

\begin{enumerate}
\item \textbf{Exp. 1 - Latencia end-to-end:} Medir latencia desde generación de telemetría en nodo IoT hasta persistencia en TimescaleDB. Variables independientes: número de nodos (N=5,10,25), frecuencia de muestreo (5s, 30s, 60s). Variables dependientes: latencia P50/P95/P99, jitter. Duración: 72 horas por configuración.

\item \textbf{Exp. 2 - Disponibilidad durante desconexión WAN:} Simular partición WAN de 48 horas desconectando Ethernet y deshabilitando LTE. Métricas: porcentaje de mensajes bufferizados exitosamente, tiempo de sincronización post-reconexión, disponibilidad de servicios locales (dashboards, alarmas).

\item \textbf{Exp. 3 - Throughput agregado HaLow:} Saturar enlace HaLow con tráfico concurrente de múltiples nodos. Medir throughput agregado vs número de clientes (N=1,5,10,20). Configuraciones: 1 MHz/2 MHz bandwidth, MCS 0-10.

\item \textbf{Exp. 4 - Failover multi-WAN:} Provocar fallas en interfaces Ethernet → HaLow → LTE. Medir tiempo de detección de falla, tiempo de conmutación, pérdida de paquetes durante transición.

\item \textbf{Exp. 5 - Overhead de procesamiento:} Caracterizar CPU/RAM/storage bajo cargas de 10/50/100 dispositivos. Identificar cuellos de botella mediante profiling (perf, flamegraphs).
\end{enumerate}

\textbf{Herramientas de medición:} Wireshark/tshark para captura de paquetes, Grafana + Prometheus para métricas de sistema, scripts Python para análisis estadístico (pandas, scipy).

\textbf{Entregables:} Datasets de mediciones (repositorio GitHub), gráficas de resultados (Capítulo 4), análisis estadístico (ANOVA, t-tests).

\subsubsection{Fase 4 - Evaluación Comparativa}

\textbf{Objetivos:} Comparar arquitectura propuesta vs soluciones baseline (cloud-centric, edge-lite).

\textbf{Baseline 1 - Cloud-Centric:} Nodos Thread → OTBR → Gateway LTE → AWS IoT Core → Lambda → DynamoDB.

\textbf{Baseline 2 - Edge-Lite:} Nodos Thread → OTBR → Node-RED (local) → AWS IoT Core (sync).

\textbf{Criterios de comparación:}
\begin{itemize}
\item Latencia P50/P99 device → storage
\item Disponibilidad durante partición WAN 48h
\item Throughput máximo (mensajes/seg)
\item Consumo energético gateway (Watts)
\item Costos OPEX (USD/mes para 100 dispositivos)
\item Complejidad de deployment (horas-persona)
\end{itemize}

\textbf{Entregables:} Tabla comparativa (Capítulo 4), análisis de trade-offs, recomendaciones de uso.

\section{Hipótesis}

\subsection{Hipótesis General}

Una arquitectura IoT para Smart Energy basada en: (1) stack de protocolos optimizado 6LoWPAN/CoAP/LwM2M sobre IEEE 802.15.4, (2) edge gateways con capacidades de procesamiento local e IA integrada, y (3) conectividad de última milla mediante IEEE 802.11ah con selección adaptativa de bandwidth (2/4/8 MHz), permite reducir la latencia end-to-end en >70\%, el overhead de paquetes en >60\%, el tráfico WAN en >65\%, garantizando disponibilidad >99\% durante desconexiones prolongadas y procesamiento inteligente en tiempo real, comparado con arquitecturas tradicionales basadas en MQTT/HTTP sobre conectividad celular.

\subsection{Hipótesis Específicas}

\textbf{H1 - Optimización mediante 6LoWPAN/CoAP/LwM2M:} La implementación del stack 6LoWPAN (compresión IPHC/NHC) + CoAP (overhead 4-10 bytes) + LwM2M (objetos binarios TLV) sobre IEEE 802.15.4 reduce el overhead de paquetes en >70\% y la latencia por salto en >40\% comparado con MQTT/JSON sobre TCP/IP, logrando tiempos de transmisión <15 ms por hop en topologías mesh de hasta 5 saltos.

\textbf{H2 - Procesamiento Edge con IA:} El despliegue de servicios containerizados edge (ThingsBoard Edge, TimescaleDB, Kafka) con integración de modelos LLM locales (Ollama + Llama 3.2 3B) permite: (a) reducción de tráfico WAN en >65\% mediante procesamiento local, (b) latencia de inferencia <500 ms para detección de anomalías, (c) disponibilidad de servicios >99\% durante desconexiones WAN >72 horas, y (d) precisión de detección de anomalías >95\% en patrones de consumo energético.

\textbf{H3 - Arquitectura Multi-Banda 802.11ah:} La arquitectura basada en gateways HaLow con selección estratégica de bandwidth según caso de uso maximiza eficiencia operacional:
\begin{itemize}
\item \textbf{2 MHz}: Óptimo para conexiones estables con sensores remotos (>2 km alcance, sensibilidad -96 dBm, tráfico <100 kbps, entornos NLOS con penetración indoor superior), logrando PDR >98\% en condiciones adversas con SNR 8-12 dB.
\item \textbf{4 MHz}: Balance ideal para gestión de red (1-1.5 km alcance, throughput 40 Mbps agregado, latencia <50 ms P95), soportando 50+ nodos con tráfico moderado (lecturas cada 15 min) sin degradación >10\%.
\item \textbf{8 MHz}: Maximiza throughput para alto tráfico con línea de vista (backhaul de concentradores, >80 Mbps, latencia <20 ms P99, alcance 0.5-1 km LOS), permitiendo agregación de datos de 100+ dispositivos por gateway.
\end{itemize}

\textbf{H4 - Compresión 6LoWPAN de Headers:} La compresión IPHC (IPv6 Header Compression) de 6LoWPAN reduce headers IPv6+UDP de 48 bytes a 2-7 bytes (compresión >85\%), y la compresión NHC (Next Header Compression) para CoAP reduce overhead adicional de 10-20 bytes a 2-4 bytes, resultando en payloads efectivos >90\% del MTU IEEE 802.15.4 (127 bytes) para aplicaciones Smart Energy.

\textbf{H5 - Eficiencia CoAP vs MQTT:} CoAP sobre UDP con modos Non-Confirmable (NON) para telemetría no crítica y Confirmable (CON) para comandos críticos, combinado con Observe para subscripciones, reduce latencia en >50\% y overhead de red en >60\% comparado con MQTT/TCP, logrando tiempos de respuesta <30 ms para transacciones GET/POST en redes Thread mesh.

\textbf{H6 - LwM2M para Gestión Eficiente:} LwM2M con objetos estándar OMA (Device, Connectivity Monitoring, Firmware Update) y transporte CoAP reduce tráfico de gestión de dispositivos en >75\% comparado con soluciones propietarias HTTP/REST, permitiendo actualizaciones OTA de firmware con transferencia block-wise sobre enlaces de baja velocidad (<250 kbps) sin timeouts.

\textbf{H7 - Procesamiento CEP Local:} El motor de reglas Complex Event Processing (CEP) de ThingsBoard Edge desplegado localmente en gateway procesa >10,000 eventos/seg con latencia <10 ms P99, ejecutando rule chains complejas (filtrado, agregación, transformación, alarmas) sin requerir round-trip WAN, reduciendo latencia de respuesta en >80\% comparado con procesamiento cloud.

\textbf{H8 - Ventaja Comparativa Integral:} La arquitectura propuesta supera a arquitecturas tradicionales (cloud-centric MQTT/LTE) en al menos 5 de 7 métricas clave: latencia (<30\% baseline), overhead paquetes (<40\% baseline), tráfico WAN (<35\% baseline), disponibilidad offline (>72h vs 0h), precisión IA (>95\% vs N/A), alcance HaLow (>150\% vs WiFi), y eficiencia energética (<60\% baseline).

\section{Objetivos}

\subsection{Objetivo General}

Diseñar, implementar y validar una arquitectura IoT centrada en edge gateways para aplicaciones Smart Energy que integre: (1) stack de protocolos optimizado 6LoWPAN/CoAP/LwM2M sobre IEEE 802.15.4 para reducción de latencia y overhead, (2) capacidades de procesamiento edge con IA local para gestión inteligente de recursos en tiempo real, y (3) conectividad de última milla mediante IEEE 802.11ah con estrategia multi-banda (2/4/8 MHz) adaptada a casos de uso específicos, garantizando latencia end-to-end <100 ms, reducción de tráfico WAN >65\%, y disponibilidad >99\% con conformidad a estándares IEEE 2030.5-2023 e ISO/IEC 30141:2024.

\subsection{Objetivos Específicos}

\textbf{OE1 - Stack de Protocolos Optimizado 6LoWPAN/CoAP/LwM2M:}
\begin{itemize}
\item Implementar capa de adaptación 6LoWPAN (RFC 6282) con compresión IPHC/NHC sobre IEEE 802.15.4, validando reducción de overhead de headers >85\% (de 48 bytes a <7 bytes) en tráfico de telemetría Smart Energy.
\item Desplegar protocolo CoAP (RFC 7252) con modos CON/NON, Observe (RFC 7641) para subscripciones, y block-wise transfer (RFC 7959), midiendo latencia <30 ms para transacciones request/response en topologías mesh 3-5 saltos.
\item Integrar LwM2M 1.2 (OMA SpecWorks) con objetos estándar (Security, Server, Device, Connectivity Monitoring, Firmware Update) para gestión unificada de dispositivos, validando reducción de tráfico de gestión >75\% vs soluciones HTTP/REST propietarias.
\item Caracterizar empíricamente PDR (Packet Delivery Ratio), latencia por hop, y consumo energético por bit transmitido en función de topología mesh (star, tree, mesh completo) y carga de red (5/10/25/50 nodos).
\end{itemize}

\textbf{OE2 - Edge Gateway con Procesamiento en Tiempo Real e IA:}
\begin{itemize}
\item Desplegar stack de servicios containerizados (ThingsBoard Edge, PostgreSQL + TimescaleDB, Apache Kafka, IEEE 2030.5 Server) sobre OpenWRT 23.05 con kernel PREEMPT\_RT, garantizando latencias de procesamiento <10 ms P99 para pipeline MQTT ingestion → rule engine → TimescaleDB persistence.
\item Integrar motor de inferencia LLM local (Ollama + Llama 3.2 3B) con latencia <500 ms para análisis de telemetría en tiempo real, implementando casos de uso: (a) detección de anomalías en consumo con precisión >95\%, (b) mantenimiento predictivo basado en patrones de alarmas, (c) compresión adaptativa de datos según bandwidth disponible.
\item Implementar gestión inteligente de recursos con adaptación dinámica: priorización de tráfico crítico (alarmas) vs no crítico (históricos), ajuste automático de frecuencia de muestreo según condiciones de red, y compactación de datos mediante CBOR/Protocol Buffers reduciendo payload >40\%.
\item Validar resiliencia mediante buffering persistente local con capacidad >100,000 mensajes (~500 MB), sincronización bidireccional post-desconexión WAN >72h con catch-up <30 minutos, y disponibilidad de servicios locales (dashboards, rule engine) >99\% durante particiones WAN.
\end{itemize}

\textbf{OE3 - Arquitectura Multi-Banda IEEE 802.11ah con Nodos HaLow:}
\begin{itemize}
\item Diseñar arquitectura de red basada en gateways edge con nodos HaLow (Morse Micro MM6108) soportando topologías Star (simple), Mesh 802.11s (auto-healing HWMP), y EasyMesh (IEEE 1905.1 roaming coordinado), validando escalabilidad a 50+ nodos por gateway sin degradación >10\% de latencia.
\item Caracterizar empíricamente desempeño por bandwidth:
  \begin{itemize}
  \item \textbf{2 MHz}: Sensibilidad -96 dBm, alcance >2 km NLOS, throughput 300-450 kbps, MCS 1-2, latencia <100 ms P95, PDR >98\% con SNR 8-12 dB. Caso de uso: sensores remotos rurales, lecturas horarias, penetración indoor.
  \item \textbf{4 MHz}: Sensibilidad -91 dBm, alcance 1-1.5 km, throughput 40 Mbps agregado, MCS 3-4, latencia <50 ms P95, soporte 50+ nodos concurrentes. Caso de uso: gestión balanceada zonas suburbanas, lecturas cada 15 min.
  \item \textbf{8 MHz}: Sensibilidad -85 dBm, alcance 0.5-1 km LOS, throughput >80 Mbps, MCS 5-7, latencia <20 ms P99. Caso de uso: backhaul de concentradores en zonas urbanas con línea de vista, agregación de 100+ dispositivos.
  \end{itemize}
\item Implementar algoritmo de selección adaptativa de bandwidth basado en: (a) condiciones de propagación (RSSI, SNR, PDR histórico), (b) requisitos de aplicación (latencia, throughput, prioridad), y (c) densidad de red (número de nodos activos, carga agregada).
\item Evaluar escalabilidad arquitectónica: topología Star (2,500 endpoints, 3 km), Mesh 802.11s (7,500 endpoints, 9 km, auto-healing <10s), EasyMesh (12,500 endpoints, roaming transparente, band steering 2/4/8 MHz).
\end{itemize}

\textbf{OE4 - Validación Experimental Comparativa:}
\begin{itemize}
\item Realizar benchmarking cuantitativo vs 2 baselines: (a) Cloud-centric (MQTT/JSON/TCP sobre LTE Cat-M1), (b) Edge-lite (Node-RED local + MQTT cloud).
\item Métricas comparadas: latencia end-to-end P50/P95/P99, overhead de paquetes (bytes header/payload), tráfico WAN (GB/mes), disponibilidad offline (horas), precisión IA (% detección correcta), alcance (km), consumo energético (mJ/bit).
\item Generar datasets públicos de mediciones (latencias, throughput, PDR) con 10+ nodos IoT ESP32-C6 Thread LwM2M en despliegue piloto de 72 horas continuas bajo condiciones variables de carga y propagación.
\end{itemize}

\textbf{OE5 - Caso de Estudio Smart Energy Real:}
\begin{itemize}
\item Desplegar prototipo funcional para 900 medidores residenciales con topología: 300 nodos ESP32-C6 Thread por gateway × 3 gateways Raspberry Pi 4 + OpenWRT + HaLow, validando arquitectura en condiciones reales urbanas/suburbanas.
\item Implementar conformidad IEEE 2030.5-2023 (Function Sets: DCAP, Time, EndDevice, MirrorUsagePoint, MirrorMeterReading, Messaging) con validación de interoperabilidad funcional vía test suite OpenADR VTN.
\item Documentar lecciones aprendidas, patrones de diseño arquitectónicos, y guías de implementación técnica (instalación OpenWRT, configuración HaLow 4 modos, despliegue stack Docker, tuning kernel PREEMPT\_RT) en anexos técnicos completos.
\end{itemize}

\section{Alcances y Limitaciones}

\subsection{Alcances}

\begin{enumerate}
\item \textbf{Diseño arquitectónico}: Especificación completa de arquitectura multi-capa con definición de componentes, interfaces y flujos de datos, mapeo a vistas ISO/IEC 30141 (funcional, información, despliegue, operacional).

\item \textbf{Implementación prototipo}: Gateway funcional basado en Banana Pi BPI-R4 con integración Thread (nRF52840 RCP), HaLow (Morse Micro MM6108), LTE (Quectel EG25-G), OpenWRT 23.05.x y stack Docker Compose con 7 servicios.

\item \textbf{Conformidad estándares}: Implementación de IEEE 2030.5-2023 Function Sets (DCAP, Time, EndDevice, MirrorUsagePoint, MirrorMeterReading, Messaging) y mapeo ISO/IEC 30141:2024.

\item \textbf{Nodos IoT}: Desarrollo de nodos ESP32-C6 Thread con cliente LwM2M, sensores BME280 y firmware actualizable OTA.

\item \textbf{Validación experimental}: Medición empírica de latencia, throughput, disponibilidad, failover y overhead en condiciones controladas de laboratorio y despliegue piloto urbano.

\item \textbf{Documentación técnica}: Anexos con guías de instalación (OpenWRT, docker-compose), configuraciones UCI completas, schemas IEEE 2030.5 XML, código fuente completo (GitHub).

\item \textbf{Evaluación comparativa}: Benchmarking cuantitativo vs 2 baselines (AWS IoT Core cloud-centric, Node-RED edge-lite) con métricas de latencia, disponibilidad, costos, complejidad.
\end{enumerate}

\subsection{Limitaciones}

\begin{enumerate}
\item \textbf{Escala de despliegue}: Validación con 10 nodos IoT y 2 gateways en área de 300 metros. No se valida escalabilidad a miles de dispositivos en despliegue real.

\item \textbf{Hardware específico}: Implementación dependiente de Morse Micro MM6108 (único chipset HaLow comercialmente disponible en 2024). Resultados pueden no generalizar a futuros chipsets.

\item \textbf{Certificación formal}: No se realiza certificación formal Thread 1.3.1 ni IEEE 2030.5. Conformidad validada mediante interoperabilidad funcional, no certificación oficial.

\item \textbf{Seguridad}: Implementación de TLS 1.2/1.3 y certificados X.509, pero sin auditoría de seguridad formal ni penetration testing exhaustivo.

\item \textbf{Estándares excluidos}: No se implementa IEC 61850 (comunicación en subestaciones) ni interoperabilidad PLC (Power Line Communication).

\item \textbf{Cobertura geográfica}: Validación en entorno urbano/suburbano. No se valida en zonas rurales remotas con cobertura celular limitada.

\item \textbf{Condiciones ambientales}: Pruebas en condiciones de laboratorio (20-25°C, humedad controlada). No se valida operación en extremos de rango industrial (-40°C a +85°C).

\item \textbf{Regulaciones RF}: Operación en banda ISM 902-928 MHz (EE.UU./América). Requiere adaptación para bandas 863-868 MHz (Europa) o 755-787 MHz (China).
\end{enumerate}

\section{Contribuciones Esperadas}

\subsection{Contribuciones Académicas}

\begin{enumerate}
\item \textbf{Arquitectura de referencia IoT heterogénea}: Especificación de arquitectura multi-capa para gateways edge que integra múltiples PHYs (802.15.4, 802.11ah, LTE), conforme con ISO/IEC 30141:2024, documentando patrones de diseño, trade-offs arquitectónicos y decisiones de ingeniería.

\item \textbf{Caracterización empírica Thread ↔ HaLow}: Primera caracterización publicada de latencias, throughput y reliability en integración Thread-HaLow mediante bridge Ethernet transparente, incluyendo análisis de overhead de OTBR y impacto de topologías mesh.

\item \textbf{Metodología IEEE 2030.5 sobre Linux embebido}: Documentación de estrategias de implementación de Function Sets IEEE 2030.5 sobre plataformas resource-constrained (ARMv8, 4 GB RAM), incluyendo optimizaciones de XML parsing, caching y gestión de certificados.

\item \textbf{Benchmarking arquitecturas edge IoT}: Dataset público de mediciones comparativas (latencia, throughput, overhead) entre arquitecturas cloud-centric, edge-lite y edge-full, proporcionando guías de selección arquitectónica basadas en requisitos de aplicación.
\end{enumerate}

\subsection{Contribuciones Técnicas}

\begin{enumerate}
\item \textbf{Implementación open-source IEEE 2030.5}: Servidor Python/Flask que implementa 6 Function Sets con schemas XML validados, autenticación TLS mutua y RBAC, disponible bajo licencia Apache 2.0 en repositorio GitHub.

\item \textbf{Configuraciones OpenWRT para HaLow}: Documentación completa de configuración UCI para driver Morse Micro MM6108 (SPI), incluyendo scripts de inicialización, configuración hostapd y troubleshooting.

\item \textbf{Stack Docker Compose optimizado}: Composición de servicios edge (ThingsBoard, TimescaleDB, Kafka, IEEE 2030.5, Ollama) con resource management, health checks y restart policies, optimizado para hardware Cortex-A53.

\item \textbf{Firmware nodos IoT Thread LwM2M}: Implementación ESP-IDF para ESP32-C6 con cliente LwM2M (Wakaama), driver BME280, Deep Sleep scheduling y OTA segura.
\end{enumerate}

\subsection{Contribuciones a la Industria}

\begin{enumerate}
\item \textbf{Reducción de costos operacionales}: Demostración de viabilidad económica de arquitectura HaLow-based vs LTE, con TCO 32\% inferior en despliegues de 1,000+ puntos durante 5 años.

\item \textbf{Guía de implementación práctica}: Documentación técnica completa (instalación, configuración, troubleshooting) que permite replicación de arquitectura por integradores de sistemas y utilities.

\item \textbf{Caso de negocio para HaLow}: Evaluación cuantitativa de beneficios (throughput, latencia, costos) de Wi-Fi HaLow vs LoRaWAN/LTE Cat-M1 en aplicaciones Smart Energy, acelerando adopción de estándar IEEE 802.11ah.

\item \textbf{Interoperabilidad multi-vendor}: Validación de conformidad IEEE 2030.5 que facilita integración con dispositivos certificados de múltiples fabricantes, reduciendo lock-in tecnológico.
\end{enumerate}

\section{Organización del Documento}

El presente documento se estructura en los siguientes capítulos:

\textbf{Capítulo 1 - Introducción}: Contextualización del problema, estado actual de tecnologías IoT, brechas identificadas, planteamiento del problema, hipótesis, objetivos, metodología, alcances y contribuciones esperadas.

\textbf{Capítulo 2 - Marco Teórico}: Fundamentos de redes Smart Energy, protocolos de comunicación IoT (Thread, HaLow, LTE Cat-M1), estándares de interoperabilidad (IEEE 2030.5, ISO/IEC 30141, IEC 61850), tecnologías de edge computing (Docker, TimescaleDB, Kafka), plataformas IoT (ThingsBoard), seguridad en sistemas IoT, y estado del arte de arquitecturas edge heterogéneas.

\textbf{Capítulo 3 - Gateway de Telemetría}: Arquitectura del gateway multi-protocolo, conformidad con estándares internacionales, requisitos funcionales/no funcionales, arquitectura jerárquica de 3 niveles IoT, diseño de hardware y software, y Stack de Servicios Containerizados.

\textbf{Capítulo 4 - Arquitectura de Telemetría}: Visión general de arquitectura end-to-end, capa de dispositivos (medidores inteligentes), capa de campo (nodos Thread, DCUs), capa de agregación (gateway HaLow), capa de aplicación (ThingsBoard cloud), análisis de seguridad end-to-end, y modelado de latencias mediante teoría de colas.

\textbf{Capítulo 5 - Conclusiones y Trabajo Futuro}: Síntesis de la investigación, cumplimiento de objetivos, validación de hipótesis, contribuciones académicas y técnicas, lecciones aprendidas, limitaciones del trabajo, y recomendaciones para trabajo futuro.

\textbf{Anexos}: Instalación OpenWRT y configuración HaLow (Anexo A), Docker Compose y servicios (Anexo B), Scripts de integración (Anexo C), Especificaciones IEEE 2030.5 (Anexo D), Implementación nodo IoT ESP32-C6 (Anexo E), Configuraciones OpenWRT UCI completas (Anexo F).

\section{Resumen del Capítulo}

Este capítulo ha establecido el contexto y la justificación de la investigación, identificando las limitaciones críticas de las arquitecturas IoT tradicionales centradas en la nube para aplicaciones de infraestructura crítica en el sector energético. Se presentó un análisis comparativo exhaustivo de las tecnologías de comunicación disponibles (Thread, Zigbee, Bluetooth Mesh para redes de campo; LoRaWAN, LTE Cat-M1, Wi-Fi HaLow para conectividad de última milla), justificando la selección de Thread y HaLow como base de la arquitectura propuesta debido a sus ventajas en términos de interoperabilidad, latencia, throughput y costos operacionales.

Se formularon cinco hipótesis cuantificables que serán validadas experimentalmente en los capítulos posteriores, abarcando aspectos de eficiencia de protocolos (H1), procesamiento edge (H2), disponibilidad operacional (H3), eficiencia energética (H4) y costo-efectividad (H5). Los objetivos específicos plantean el diseño, implementación, validación experimental y evaluación comparativa de una arquitectura IoT jerárquica de tres niveles (nodos, routers, gateways) con cumplimiento de estándares internacionales IEEE 2030.5 e ISO/IEC 30141.

Las contribuciones esperadas del trabajo abarcan tres dimensiones: académicas (caracterización empírica Thread-HaLow, benchmarking de arquitecturas edge), técnicas (implementaciones open-source, configuraciones OpenWRT, firmware IoT) e industriales (reducción de costos operacionales, guías de implementación práctica, casos de negocio para adopción de HaLow).

El siguiente capítulo (Marco Teórico) profundiza en los fundamentos teóricos de las tecnologías seleccionadas, presentando el estado del arte de los protocolos de comunicación IoT, los estándares de interoperabilidad para Smart Energy y las plataformas de procesamiento en el borde, estableciendo las bases conceptuales para el diseño de la arquitectura propuesta que se detalla en el Capítulo 3.
