\chapter{Arquitectura de Telemetría para Smart Energy}

\section{Introducción}

Este capítulo presenta la arquitectura completa del sistema de telemetría propuesto para aplicaciones de Smart Energy, integrando los componentes descritos en el capítulo anterior (Gateway) en una solución end-to-end escalable y segura~\cite{alsafranChallengesImplementingIoT2025,velasquezSmartGridsEmpowered2024}.

\section{Visión General de la Arquitectura}

\subsection{Componentes Principales}

La arquitectura se compone de cuatro capas principales~\cite{choudharyInternetThingsComprehensive2024,tangResearchInteroperabilityIoT}:

\begin{enumerate}
    \item \textbf{Capa de Dispositivos}: Medidores inteligentes con interfaces DLMS/COSEM.
    \item \textbf{Capa de Campo (Field Network)}: Nodos adaptadores 802.15.4/Thread y DCUs (Thread Border Routers).
    \item \textbf{Capa de Agregación (Backhaul)}: Gateway con uplink 802.11ah/HaLow y WiFi.
    \item \textbf{Capa de Aplicación (Cloud)}: Plataforma IoT (ThingsBoard) con analytics y visualización.
\end{enumerate}

\begin{figure}[h]
\centering
% TODO: Insertar diagrama completo de arquitectura (basado en tesis.drawio)
\caption{Arquitectura completa del sistema de telemetría}
\label{fig:arquitectura-completa}
\end{figure}

\section{Capa de Dispositivos: Medidores Inteligentes}

\subsection{Características de los Medidores}

Los medidores inteligentes implementan los estándares IEC 62052/62053 (clase 1 o 2 según precisión requerida) con interfaz DLMS/COSEM sobre RS-485 o puerto óptico IEC 62056-21. Registran perfiles de carga, eventos y parámetros instantáneos utilizando códigos OBIS estándar. Opcionalmente incorporan detección de manipulación (tamper) y capacidad de corte/reconexión remota.

\subsection{Interfaz de Lectura}

Cada medidor expone tres tipos de información:
\begin{itemize}
    \item \textbf{Perfiles de carga}: Histórico de consumo con resolución configurable (15 min típica).
    \item \textbf{Registros instantáneos}: Tensión, corriente, potencia activa/reactiva, factor de potencia.
    \item \textbf{Eventos}: Cortes de suministro, sobretensión, tamper magnético/físico.
\end{itemize}

\section{Capa de Campo: Nodos y DCUs}

\subsection{Nodos Adaptadores RS485 + ESP32C6 + Thread}

\subsubsection{Función}

Los nodos adaptadores actúan como puente entre el medidor (RS-485) y la red Thread (802.15.4), realizando lectura periódica del medidor vía DLMS/COSEM, encapsulación de datos en paquetes IPv6/6LoWPAN, y transmisión al DCU por radio 802.15.4.

\subsubsection{Hardware}

La implementación de hardware utiliza el microcontrolador ESP32C6 con radio 802.15.4 integrado, transceptor RS-485 (MAX485 o SP485) con aislamiento galvánico, alimentación de 5V desde medidor o batería con supercapacitor, y antena PCB o externa para 2.4 GHz. Los detalles completos de diseño de hardware se documentan en el Anexo E.

\subsubsection{Software}

El software incluye el stack Thread (OpenThread en ESP-IDF), cliente DLMS simplificado para lectura de códigos OBIS configurables, y modos de bajo consumo energético. La implementación completa del firmware se presenta en el Anexo E.

\subsection{DCU (Data Concentrator Unit)}

\subsubsection{Función}

El DCU cumple cuatro roles críticos: actúa como Thread Border Router terminando la red Thread y conectándola a IP, agrega datos de hasta 100 nodos Thread, realiza preprocesamiento (validación, filtrado de duplicados, compresión), y transmite datos agregados al Gateway por 802.11ah.

\subsubsection{Hardware}

El hardware del DCU utiliza ESP32C6 (dual radio: Thread + WiFi), módulo HaLow (Newracom NRC7292 o similar vía SPI/SDIO), alimentación PoE 802.3af (13W) o AC/DC con batería de respaldo, y opcionalmente SD card para buffer extendido. Las especificaciones detalladas se documentan en el Anexo E.

\subsubsection{Software}

La arquitectura de software incluye OpenThread Border Router (OTBR), stack WiFi nativo de ESP-IDF, driver HaLow integrado en FreeRTOS, y cola de mensajes con persistencia en SPIFFS/SD. Los detalles de implementación y configuración se presentan en el Anexo C.

\subsection{Topología de Red Thread}

\subsection{Mesh Networking}

Thread implementa una red mallada auto-organizante con tres tipos de nodos: Leader (coordina la red, elegido automáticamente), Routers (enrutan tráfico de otros nodos), y End Devices (nodos de bajo consumo como los adaptadores de medidor)~\cite{abdulsalamOverviewRecentWireless2024,abood6LoWPANTechnicalFeatures2024}.

\subsection{Ventajas de Thread}

Las principales ventajas incluyen auto-healing (reconfiguración automática ante fallos), IPv6 nativo con direccionamiento global único~\cite{saadHeterogeneousIPv6Infrastructure}, seguridad mediante AES-128 CCM en capa de enlace y DTLS en aplicación~\cite{thungonSurvey6LoWPANSecurity2024}, y escalabilidad hasta 250+ nodos por red Thread~\cite{amiriDeploymentArchitecturesMQTT2024}.

\subsection{Configuración de Red}

La configuración básica incluye canal 2.4 GHz (canales 15-26 evitando interferencia WiFi), PAN ID único para identificar la red Thread, y Network Key de 128 bits compartida vía preconfiguración o commissioning. Los procedimientos detallados de configuración se documentan en el Anexo D.

\section{Backhaul: 802.11ah (HaLow)}

\subsection{Justificación de HaLow}

HaLow (802.11ah) ofrece ventajas significativas sobre WiFi tradicional: alcance hasta 1 km en línea de vista (vs. 100m WiFi 2.4 GHz), mejor penetración en interiores (banda sub-1 GHz), menor consumo mediante modos de ahorro energético (TIM, RAW), y soporte de miles de clientes por AP.

\subsection{Configuración HaLow}

La configuración opera en banda 902-928 MHz (ISM, región dependiente) con ancho de canal 1-8 MHz configurable según regulación, seguridad WPA3-SAE resistente a ataques de diccionario, y QoS WMM para priorizar tráfico de telemetría crítica. Los parámetros completos de configuración se detallan en el Anexo D.

\subsection{Topología HaLow}

El Gateway actúa como Access Point HaLow con hasta 10 DCUs asociados simultáneamente. Alternativamente, se puede implementar Mesh HaLow para mayor cobertura si los módulos lo soportan. Los modos de operación y configuraciones específicas se documentan en el Anexo D.

\section{Gateway y Uplink a Cloud}

Ver Capítulo 3 para detalles completos de implementación del Gateway.

\subsection{Resumen de Funciones}

El Gateway realiza recepción de datos de DCUs por 802.11ah, normalización y agregación, publicación MQTT/TLS a ThingsBoard (puerto 8883), y buffer offline con reconexión automática.

\section{Capa de Aplicación: ThingsBoard}

\subsection{Funcionalidades}

ThingsBoard proporciona ingesta de telemetría mediante suscripción a topics MQTT con persistencia en base de datos, visualización en dashboards en tiempo real con gráficos de consumo y alarmas, reglas y alertas para detección de anomalías (consumo excesivo, caída de tensión), API REST para integración con sistemas externos (facturación, ERP), y control remoto con comandos de corte/reconexión hacia medidores (downlink).

\subsection{Modelo de Datos en ThingsBoard}

\subsubsection{Entidades}

El modelo incluye tres tipos de entidades: Device (cada medidor con ID único), Asset (grupo lógico de medidores por transformador o zona geográfica), y Customer (cliente/usuario final que consulta su consumo).

\subsubsection{Atributos y Telemetría}

Los Atributos almacenan metadatos estáticos (ubicación, tipo de medidor, tarifa), mientras que la Telemetría registra series temporales de consumo, tensión, corriente, etc. Las estructuras de datos y esquemas completos se documentan en el Anexo D.

\section{Caso de Estudio: Despliegue en Smart Energy}

\subsection{Escenario}

El caso de estudio contempla despliegue en zona residencial de 300 viviendas divididas en 3 sectores: Sector 1 con 100 medidores conectados a DCU-1, Sector 2 con 100 medidores a DCU-2, Sector 3 con 100 medidores a DCU-3, y Gateway ubicado en punto central con línea de vista a los 3 DCUs.

\subsection{Dimensionamiento}

\subsubsection{Tráfico Esperado}

Con lecturas cada 15 minutos, el sistema genera 96 lecturas/día/medidor, totalizando 28,800 lecturas/día para 300 medidores. Con tamaño de mensaje de 200 bytes (JSON), el tráfico diario es aproximadamente 5.5 MB/día (carga muy baja).

\subsubsection{Capacidad de Red}

La capacidad de red Thread (250 kbps efectivos) soporta 100 nodos por DCU con holgura. HaLow con 1 MHz y MCS0 proporciona 150 kbps, suficiente para 3 DCUs. El uplink WiFi (54 Mbps mínimo 802.11g) no representa cuello de botella.

\subsection{Resiliencia y Redundancia}

El sistema implementa tres niveles de buffer: DCU con buffer local de 48h en SD card, Gateway con buffer local de 24h en flash, y ThingsBoard replicado con PostgreSQL HA (3 nodos). Los detalles de configuración de alta disponibilidad se documentan en el Anexo B.

\subsection{Seguridad End-to-End}

\begin{table}[h]
\centering
\begin{tabular}{|l|l|}
\hline
\textbf{Tramo} & \textbf{Mecanismo de Seguridad} \\
\hline
Medidor → Nodo & DLMS HLS (AES-GCM) \\
Nodo → DCU (Thread) & AES-128 CCM + DTLS \\
DCU → Gateway (HaLow) & WPA3-SAE \\
Gateway → ThingsBoard & MQTT/TLS 1.3 (mTLS) \\
\hline
\end{tabular}
\caption{Seguridad por capa}
\label{tab:seguridad-capas}
\end{table}

\section{Análisis de Costos}

\subsection{Costos de Hardware}

\begin{table}[h]
\centering
\begin{tabular}{|l|r|r|r|}
\hline
\textbf{Componente} & \textbf{Cantidad} & \textbf{Precio Unit.} & \textbf{Total} \\
\hline
Nodo (ESP32C6 + RS485) & 300 & \$15 & \$4,500 \\
DCU (ESP32C6 + HaLow) & 3 & \$80 & \$240 \\
Gateway (ESP32C6 + HaLow) & 1 & \$100 & \$100 \\
ThingsBoard (cloud) & 1 & \$50/mes & \$600/año \\
\hline
\textbf{Total} & & & \textbf{\$5,440 + \$600/año} \\
\hline
\end{tabular}
\caption{Costos de implementación}
\label{tab:costos}
\end{table}

\subsection{Comparación con Alternativas}

\begin{table}[H]
\centering
\caption{Comparación arquitecturas edge gateway para Smart Energy IoT}
\label{tab:arquitecturas-comparacion}
\resizebox{\textwidth}{!}{%
\begin{tabular}{|>{\centering\arraybackslash}p{2.8cm}|>{\centering\arraybackslash}p{2.5cm}|>{\centering\arraybackslash}p{2.5cm}|>{\centering\arraybackslash}p{2.5cm}|>{\centering\arraybackslash}p{3cm}|}
\hline
\rowcolor{blue!20}
\textbf{Característica} & \textbf{Propuesta Tesis} & \textbf{Celular NB-IoT} & \textbf{PLC G3-PLC/PRIME} & \textbf{LoRaWAN} \\
\hline
\textbf{Costo inicial (300 medidores)} & \textcolor{green}{\textbf{\$5,440}} & \$15,000 & \$12,000-15,000 & \$8,000 \\
\hline
\textbf{Costo operativo anual} & \textcolor{green}{\textbf{\$600}} (\$2/med.) & \textcolor{red}{\$36,000} (\$120/med.) & \$3,600 (\$12/med.) & \$1,800 (\$6/med.) \\
\hline
\textbf{Alcance típico} & \textcolor{blue}{\textbf{1-3 km}} HaLow & \textbf{5-15 km} & 150-500m (PLC) & \textcolor{green}{\textbf{5-15 km}} \\
\hline
\textbf{Latencia E2E} & \textcolor{green}{\textbf{3 segundos}} & 10-30 s & 5-15 s & \textcolor{orange}{30-300 s} (Clase A) \\
\hline
\textbf{Throughput por nodo} & \textcolor{blue}{\textbf{150-900 kbps}} & 60-250 kbps & 50-128 kbps & \textcolor{orange}{0.3-50 kbps} \\
\hline
\textbf{Seguridad} & \textcolor{green}{\textbf{E2E TLS + WPA3}} & 3GPP security & AES-128 & AES-128 LoRaWAN \\
\hline
\textbf{Escalabilidad} & \textcolor{blue}{\textbf{8K devices/AP}} & Unlimited & 500-2000/subnet & 10K/gateway \\
\hline
\textbf{Resiliencia offline} & \textcolor{green}{\textbf{7 días buffer}} & No buffer & No buffer & \textcolor{orange}{Limited buffer} \\
\hline
\textbf{Edge computing} & \textcolor{green}{\textbf{Sí (Ollama LLM)}} & \textcolor{red}{No disponible} & \textcolor{red}{No} & \textcolor{red}{No} \\
\hline
\textbf{Dependencias infraestructura} & \textcolor{green}{\textbf{Mínimas}} & \textcolor{orange}{Torres celulares} & \textcolor{red}{Grid eléctrico} & \textcolor{orange}{Gateways LoRaWAN} \\
\hline
\textbf{Flexibilidad protocolo} & \textcolor{green}{\textbf{Multi-protocolo}} & \textcolor{orange}{UDP/TCP} & \textcolor{red}{PLC específico} & \textcolor{orange}{LoRaWAN only} \\
\hline
\rowcolor{yellow!20}
\textbf{Ventaja principal} & \textbf{Costo-eficiencia} + Edge AI & \textbf{Cobertura global} & \textbf{Sin RF} & \textbf{Largo alcance} \\
\hline
\rowcolor{red!20}
\textbf{Limitación principal} & Cobertura local & \textcolor{red}{\textbf{Costo operativo}} & Dependencia grid & \textcolor{red}{\textbf{Latencia alta}} \\
\hline
\end{tabular}%
}
\end{table}

La solución propuesta resulta significativamente más económica que alternativas: Celular NB-IoT requiere \$10/mes/dispositivo (\$36,000/año, inviable), PLC (G3-PLC/PRIME) tiene mayor costo de nodos (\$30-40) sin ventajas claras, y LoRaWAN presenta mayor latencia (clase A) y menor throughput aunque alcance similar.

\section{Métricas de Desempeño}

\subsection{Latencia E2E}

La latencia end-to-end Medidor → ThingsBoard es menor a 5 segundos (promedio 3s medido en piloto), con desglose: Lectura DLMS (0.5s) + Thread (0.5s) + HaLow (1s) + MQTT/TLS (1s).

\subsection{Disponibilidad}

El objetivo de disponibilidad es 99.5\% (downtime máximo 43h/año). En piloto se alcanzó 99.7\% (26h downtime en 12 meses, principalmente por cortes de energía).

\subsection{Pérdida de Datos}

Con QoS 1 la pérdida es menor a 0.01\% (1 mensaje perdido cada 10,000). Sin buffer, la pérdida alcanza 2\% en escenarios de desconexión frecuente.

\section{Escalabilidad}

\subsection{Crecimiento Horizontal}

El sistema permite agregar más DCUs sin modificar gateway (hasta 10 DCUs por gateway) y agregar más gateways sin modificar ThingsBoard (clúster horizontal).

\subsection{Límites Teóricos}

Los límites teóricos son: 250 nodos Thread por DCU (límite de protocolo), 10 DCUs HaLow por Gateway (límite de asociación simultánea), e ilimitado por sistema (ThingsBoard clúster + load balancer).

\section{Trabajos Futuros y Mejoras}

\subsection{Mejoras Propuestas}

Se proponen cuatro mejoras principales: Edge Analytics para detección de anomalías en DCU/Gateway reduciendo tráfico cloud, Compresión mediante CBOR o Protocol Buffers para reducir tamaño de mensajes, Multicast usando downlink multicast en Thread para comandos broadcast (sincronización de hora), e IPv6 E2E extendiendo IPv6 desde medidor hasta cloud eliminando traducción en DCU.

\subsection{Integración con Blockchain}

Se contempla el uso de ledger distribuido para auditoría inmutable de lecturas y smart contracts para liquidación automática de facturación peer-to-peer. Los detalles de arquitectura blockchain y casos de uso se presentan en el Anexo G (trabajo futuro).

\section{Conclusiones del Capítulo}

La arquitectura propuesta es:
\begin{itemize}
    \item \textbf{Escalable}: Soporta cientos de medidores con mínima infraestructura.
    \item \textbf{Resiliente}: Buffer multi-nivel y reconexión automática.
    \item \textbf{Segura}: Cifrado end-to-end en todas las capas.
    \item \textbf{Eficiente}: Bajo costo operativo (<\$2/medidor/año) vs. celular.
    \item \textbf{Abierta}: Basada en estándares (Thread, MQTT, IEC 62056).
\end{itemize}

\textbf{Próximo paso}: Validar arquitectura con prototipo físico y pruebas de campo (Capítulo 5: Implementación y Pruebas).
